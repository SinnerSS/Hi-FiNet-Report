Wireless Sensor Networks (WSN) are widely used in a variety of fields such as healthcare, logistics, military, and envinronment monitoring. The rapid advancement in Micro-Electro-Mechanical Systems (MEMS) technology has enabled the creation of smart sensors that are limited in computation, energy and memory but are significantly more cost efficient than traditional sensors \cite{Yick2008, Chai2020, Hussain2021}. These networks can be deployed to provide real-time updates on temperature, humidity, noise and other properties of the environment \cite{Yick2008, Chai2020, Ullo2020}.

Because of their proliferation, environments in which WSNs are deployed are also very diverse, including highly dangerous, inaccessible environments \cite{Prasad2023}. Therefore, the data sent back to the base station can be erroneous, either from hardware, software, communication issues, or any combination of the above. Depending on the angle of view, there are two main ways to categorize faults in the data: time-based and characteristic-based. From a time-based perspective, faults can be classified as soft permanent, intermittent, and transient faults \cite{Prasad2023}. According on the literature, characteristic-based fault types can vary, but often include offset, stuck-at and hardover faults with some additional types \cite{Shi2024,Saeed2021, Ni2009}. Understanding these fault types informs the design of the detection algorithms, which we categorize next.

Existing fault detection methods in WSNs include model-based approaches, data-driven approaches and hybrid information-based methods \cite{Shi2024}. The most common approach is a model-based algorithm, which utilize mathematical and statistical principles to model each fault type. The data-driven approach on the other hand uses the analysis of data samples obtained to build a model for fault classification. Hybrid information-based methods use both human knowledge and data through a combination of different methods. These algorithms can be further classified based on the input data used into: centralized, distributed and self-diagnosis methods \cite{Prasad2023, Takele2024}. Centralized fault classification collects information from the entire network and processes at a central location (usually the base station). Distributed methods offer more scalibility by dividing the network into cluster of sensors with a cluster leader responsible for fault classficication of its cluster. Self-diagnosis approachs requires each node to determine its own fault status based on only data collected from its sensors. Prior work either focuses on temporal patterns at individual nodes or spatial relations across neighbors, but fails to integrate both effectively. In this work, we investigate the efficacy of a data-driven hierarchical method for WSNs fault diagnosis. The major contribution of this paper are as follows:

\begin{itemize} 
  \item We propose HiFiNet: a hierarchical network utilizing an Long Short-Tern Memory (LSTM) classifier for edge-based phase and a Graph Attention Network (GAT) for neighbor-aggregation phase. 
  \item Dataset composed of fault injected samples with five different fault types based on Intel Lab Dataset. 
  \item Results include various metrics such as: accuracy, F1-score, recall on the aforementioned dataset, demonstrate improvement against methods in the literature \end{itemize}
